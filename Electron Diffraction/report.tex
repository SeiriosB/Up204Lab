\documentclass[12pt, a4paper]{article}
\usepackage{graphicx}
\usepackage{amsmath}
\usepackage{geometry}
\usepackage{float}
\usepackage{booktabs}
\usepackage{caption}

% Page Setup
\geometry{margin=1in}

\begin{document}

% --- Title Page ---
\begin{titlepage}
    \centering
    \vspace*{\fill}
    
    {\Huge \textbf{Electron Diffraction}\par}
    
    \vspace{1.5cm}
    
    {\Large Animesh Pradhan \\ 24551\par}
    
    \vspace{1.5cm}
    
    {\large \today\par}
    
    \vspace*{\fill}
\end{titlepage}

\section{Brief Introduction}

This experiment employs electron diffraction to ascertain the lattice parameters of a polycrystalline graphite sample and to validate the de Broglie hypothesis. De Broglie proposed that matter possesses wave-like properties, with a wavelength defined by:

\begin{equation*}
    \lambda = \frac{h}{mv}
\end{equation*}

When an electron is accelerated across a potential difference $V$, its wavelength is described as:

\begin{equation*}
    \lambda = \frac{h}{\sqrt{2mVe}}
\end{equation*}

To test this hypothesis, we apply Bragg's law of diffraction:

\begin{equation*}
    2d \sin \theta = n\lambda
\end{equation*}

The diffraction angle $\theta$ is derived by measuring the diameter of the observed concentric rings. Based on the geometric arrangement, this is calculated as:

\begin{equation*}
    \theta = \frac{1}{2} \tan^{-1} \left(\frac{D}{2L}\right)
\end{equation*}

\section{Prelab}

\begin{figure}[H]
    \centering
    \includegraphics[width=0.9\textwidth]{/home/Animesh/phys_lab_sem4/up204/Electron Diffraction/prelab/1.jpg}
\end{figure}
\begin{figure}[H]
    \centering
    \includegraphics[width=0.9\textwidth]{/home/Animesh/phys_lab_sem4/up204/Electron Diffraction/prelab/2.jpg}
\end{figure}

\subsection{Data}

\begin{figure}[H]
    \centering
    \includegraphics[width=0.9\textwidth]{/home/Animesh/phys_lab_sem4/up204/Electron Diffraction/data/data.jpg}
\end{figure}

\section{Graphs}

\begin{figure}[H]
    \centering
    \includegraphics[width=0.9\textwidth]{/home/Animesh/phys_lab_sem4/up204/Electron Diffraction/graph/lambda_theo_plot.png}
    \caption{Plot of $\lambda_{theo}$ vs $\sin(\theta_2)$}
\end{figure}

\begin{figure}[H]
    \centering
    \includegraphics[width=0.9\textwidth]{/home/Animesh/phys_lab_sem4/up204/Electron Diffraction/graph/lambda_exp_plot.png}
    \caption{Plot of $\lambda_{exp}$ vs $\sin(\theta_2)$}
\end{figure}

\section{Analysis}

\subsection{Verification of de Broglie Hypothesis}
To verify the hypothesis, we compare the theoretical de Broglie wavelength ($\lambda_{theo}$) with the experimental wavelength ($\lambda_{exp}$) derived from the diffraction pattern of the first ring ($D_1$).

The theoretical wavelength is calculated using the accelerating voltage $V$:
\begin{equation}
    \lambda_{theo} = \frac{h}{\sqrt{2m_e e V}} \approx \frac{12.27}{\sqrt{V_{k}}} \, \text{\AA}
\end{equation}

The experimental wavelength is determined using Bragg's Law:
\begin{equation}
    \lambda_{exp} = 2d_1 \sin(\theta_1)
\end{equation}
where $d_1 = 2.13 \, \text{\AA}$ (inter-planar distance for graphite) and the diffraction angle $\theta_1$ is derived from the ring radius $r_1$ and the distance $L = 14.0 \, \text{cm}$:
\begin{equation}
    \theta_1 = \frac{1}{2} \tan^{-1}\left(\frac{r_1}{L}\right)
\end{equation}

The results are summarized in Table \ref{tab:verification}.

\begin{table}[h!]
    \centering
    \caption{Experimental Verification of de Broglie Hypothesis}
    \label{tab:verification}
    \begin{tabular}{ccccccc}
        \toprule
        Voltage & $r_1$ & $\lambda_{theo}$ & $\theta_1$ & $\sin(\theta_1)$ & $\lambda_{exp}$ & \% Error \\
        (kV) & (mm) & (\AA) & (rad) & & (\AA) & \\
        \midrule
        2.0 & 17.54 & 0.274 & 0.0623 & 0.0623 & 0.265 & 3.28\% \\
        2.5 & 15.81 & 0.245 & 0.0562 & 0.0562 & 0.239 & 2.45\% \\
        3.0 & 14.00 & 0.224 & 0.0498 & 0.0498 & 0.212 & 5.35\% \\
        3.5 & 13.02 & 0.207 & 0.0463 & 0.0463 & 0.197 & 4.83\% \\
        4.0 & 12.27 & 0.194 & 0.0437 & 0.0437 & 0.186 & 4.12\% \\
        4.5 & 11.80 & 0.183 & 0.0420 & 0.0420 & 0.179 & 2.19\% \\
        5.0 & 11.36 & 0.173 & 0.0405 & 0.0405 & 0.172 & 0.58\% \\
        \bottomrule
    \end{tabular}
\end{table}

Clearly, the experimental wavelengths closely match the theoretical predictions, with percentage errors below 6\%, thus confirming the de Broglie hypothesis.

\subsection{Determination of Inter-planar Distance $d_2$}
The second diffraction ring corresponds to an unknown inter-planar distance $d_2$. According to Bragg's law ($n\lambda = 2d_2 \sin \theta_2$), a plot of $\lambda$ versus $\sin(\theta_2)$ yields a straight line with a slope equal to $2d_2$.

The data for the second ring is compiled in Table \ref{tab:d2_data}, with $\lambda_{theo}$ and $\lambda_{exp}$ as before.
\begin{table}[H]
    \centering
    \caption{Data for Determination of $d_2$}
    \label{tab:d2_data}
    \begin{tabular}{cccccc}
        \toprule
        Voltage (kV) & $r_2$ (mm) & $\theta_2$ (rad) & $\sin(\theta_2)$ & $\lambda_{theo}$ (\AA) & $\lambda_{exp}$ (\AA) \\
        \midrule
        2.0 & 29.23 & 0.1029 & 0.1027 & 0.274 & 0.265 \\
        2.5 & 28.08 & 0.0989 & 0.0988 & 0.245 & 0.239 \\
        3.0 & 25.00 & 0.0883 & 0.0882 & 0.224 & 0.212 \\
        3.5 & 24.16 & 0.0854 & 0.0853 & 0.207 & 0.197 \\
        4.0 & 21.38 & 0.0758 & 0.0757 & 0.194 & 0.186 \\
        4.5 & 20.98 & 0.0744 & 0.0743 & 0.183 & 0.179 \\
        5.0 & 19.90 & 0.0706 & 0.0705 & 0.173 & 0.172 \\
        \bottomrule
    \end{tabular}
\end{table}

The plots of $\lambda$ versus $\sin(\theta_2)$ are shown in the Graphs section. According to Bragg's law, the relationship $\lambda = 2d_2 \sin(\theta_2)$ implies that a linear fit should yield a slope equal to $2d_2$.

\subsubsection{Analysis Using Theoretical Wavelength}
From the linear regression of $\lambda_{theo}$ versus $\sin(\theta_2)$, we obtain:
\begin{itemize}
    \item Slope: $(2.8444 \pm 0.2397)$ \AA
    \item Intercept: $-0.0277$ \AA
    \item $R^2 = 0.9657$
\end{itemize}

The inter-planar distance is determined from the slope:
\begin{equation*}
    d_2 = \frac{\text{slope}}{2} = \frac{2.8444}{2} = (1.422 \pm 0.120) \, \text{\AA}
\end{equation*}

\subsubsection{Analysis Using Experimental Wavelength}
From the linear regression of $\lambda_{exp}$ versus $\sin(\theta_2)$, we obtain:
\begin{itemize}
    \item Slope: $(2.6842 \pm 0.2532)$ \AA
    \item Intercept: $-0.0212$ \AA
    \item $R^2 = 0.9574$
\end{itemize}

The inter-planar distance is:
\begin{equation*}
    d_2 = \frac{\text{slope}}{2} = \frac{2.6842}{2} = (1.342 \pm 0.127) \, \text{\AA}
\end{equation*}

Both analyses yield high $R^2$ values ($>0.95$), confirming excellent linear fits and validating Bragg's law. 

The average inter-planar distance from both methods is:
\begin{equation*}
    d_2 = \frac{1.422 + 1.342}{2} = 1.382 \, \text{\AA}
\end{equation*}
The uncertainty is calculated as:
\begin{equation*}
    \Delta d_2 = \frac{0.120 + 0.127}{2} = 0.124 \, \text{\AA}
\end{equation*}
Thus, we have:
\begin{equation*}
    d_2 = (1.382 \pm 0.124) \, \text{\AA}
\end{equation*}

The known inter-planar spacing for the (1120) plane of graphite is approximately 1.23 \AA, which is reasonably close to our experimental results considering the uncertainty.

\subsection{Error Analysis}

The uncertainty in the experimental wavelength, $\Delta \lambda_{exp}$, is calculated using the propagation of errors from the measurement of the diffraction ring radius $r$.

The relationship for wavelength is given by Bragg's Law:
\begin{equation}
    \lambda = 2 d_1 \sin \theta
\end{equation}

\begin{equation}
    \Rightarrow \Delta \lambda = 2 d_1 \cos \theta \cdot \Delta \theta
\end{equation}

The diffraction angle $\theta$ is determined by the geometry of the setup:
\begin{equation}
    \theta = \frac{1}{2} \tan^{-1}\left(\frac{r}{L}\right)
\end{equation}
The uncertainty in $\theta$, denoted as $\Delta \theta$, depends on the uncertainty in the radius measurement ($\Delta r$) and the distance $L$. Assuming $L$ is constant relative to the variation in $r$:
\begin{equation}
    \Delta \theta \approx \frac{1}{2} \frac{1}{1 + (r/L)^2} \frac{\Delta r}{L}
\end{equation}
Here, $\Delta r$ is estimated as half the width of the diffraction ring (calculated from the spread in the raw inner/outer measurements).

Table \ref{tab:error_analysis} summarizes the uncertainty calculations. The theoretical wavelength $\lambda_{theo}$ generally falls within the range $\lambda_{exp} \pm \Delta \lambda$, indicating the deviations are largely accounted for by measurement limitations.

\begin{table}[h!]
    \centering
    \caption{Uncertainty Analysis for $\lambda_{exp}$}
    \label{tab:error_analysis}
    \begin{tabular}{cccccc}
        \toprule
        Voltage & $\Delta r$ (est) & $\Delta \theta$ & $\lambda_{exp}$ & $\Delta \lambda$ & $\lambda_{theo}$ \\
        (kV) & (mm) & (rad) & (\AA) & (\AA) & (\AA) \\
        \midrule
        2.0 & 2.45 & 0.0087 & 0.265 & 0.037 & 0.274 \\
        2.5 & 1.29 & 0.0046 & 0.239 & 0.020 & 0.245 \\
        3.0 & 1.55 & 0.0055 & 0.212 & 0.024 & 0.224 \\
        3.5 & 1.43 & 0.0051 & 0.197 & 0.022 & 0.207 \\
        4.0 & 0.80 & 0.0029 & 0.186 & 0.012 & 0.194 \\
        4.5 & 0.81 & 0.0029 & 0.179 & 0.012 & 0.183 \\
        5.0 & 0.72 & 0.0026 & 0.172 & 0.011 & 0.173 \\
        \bottomrule
    \end{tabular}
\end{table}

\section{Conclusion}

\begin{itemize}
    \item The experiment successfully verified the de Broglie hypothesis, with experimental wavelengths closely matching theoretical predictions within a 6\% error margin.
    \item The inter-planar distance $d_2$ for the second diffraction ring was determined to be approximately $1.382 \pm 0.124$ \AA, quite close to the known value of $1.23$ \AA\ for graphite.
    \item Uncertainties in the experimental values of wavelengths were determined, with the theoretical values falling within the calculated error bounds. 
    \item Measurement uncertainties, particularly in ring diameter and path length, were the primary sources of error.
\end{itemize}

\end{document}
